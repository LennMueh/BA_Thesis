For the Spectrum-based fault localization, we adapted the DeepFault \cite{eniser_deepfault_2023} code to our needs.
Especially, we adapted the functions to the recent versions of the used libraries and added some additional functions to save and load the data.

The functions adapted from the DeepFault paper are now invoked in the \texttt{run\_analysis} function seen in listing \ref{lst:main analysis}, which is the main function for the analysis.
We have implemented it that way, to ease the use of the analysis functions, in comparison to the original DeepFault code, which was just one large python script.
The function manages the loading of the model and the experiment or working paths.
It checks if the necessary steps of the analysis have already been executed, hence the data is already available out of a previous run and uses them instead, because we assume that the model has not changed, and the data is still valid.

The data we need are the classifications, the layer outputs and the spectrum matrices.
The classifications are the indexes of the data of the test dataset sorted in to the categories of correctly predicted and not correctly predicted.
The layer outputs are the information about the activation for each neuron, of each layer, for each member of the test dataset.
Now with this information we can construct the spectrum matrix, which uses the layer outputs and classifications to determine sums of the symbols denoted in the table \ref{tab: Sus symbols}, for each neuron, if it was activated or not, and if the prediction of the network was right or wrong.

When all this data is aggregated, we can determine the suspiciousness of each neuron, which is the main goal of the analysis.
All this data is then saved to the working path, for future runs, and the suspicious neurons are returned.
The network's neuron count is aggregated, and if the amount of selected neurons \texttt{susp\_num} is less than the amount of modifiable neurons, the \texttt{susp\_num} most suspicious neurons are returned.

Otherwise, all modifiable neurons are returned, ranked by their level of suspicion.
To determine the suspiciousness, we use the $\text{D}^*$, Tarantula, and Ochiai methods, which you can see in the background chapter \ref{sec:spectrum-based-fault-localisation}.
We also added the option to return a specified amount of random neurons, which can be useful for testing purposes, you can see the function in listing \ref{lst:random_choosing}.
The function returns a list of tuples in the format of $(layer, layer\_index)$, ranked from the most suspicious neuron to the least suspicious.

\lstinputlisting[language=Python, linerange={8-53,65}, caption={Main analysis function}, label={lst:main analysis}]{source_analysis/__init__.py}
\lstinputlisting[language=Python, linerange={59-71}, caption={Random Choosing, of neurons}, label={lst:random_choosing}]{source_analysis/analysis.py}

The saving and loading functions in listings \ref{lst:saving} required adaptation due to the depreciation of the \texttt{.value} property in h5py \cite{collette_h5pyh5py_2022}.
Therefore, every loading function needs to use indexing instead of the property used by the old DeepFault code.
Save and load functions were added for the spectrum matrices and ranked neurons, for saving time by not having to re-run the analysis every time the program is executed.
Because of the similarity of the functions, we only show the saving and load function for the classifications in listing \ref{lst:saving}.
All functions can be found in the appendix \ref{lst:saving_full}.
\lstinputlisting[language=Python, linerange={19-43}, caption={Saving and loading the classifications}, label={lst:saving}]{source_analysis/utilities.py}

Additionally, we modified the \texttt{get\_layer\_outs} function as shown in listing \ref{lst:layer outs} from native Keras \cite{chollet_keras_2015} calls to TensorFlow \cite{martin_abadi_tensorflow_2015} via \texttt{tf.keras}, which uses Keras as its backend, as recommended from Keras versions 2.3.0. \cite{keras_release_nodate} onward.
Because of its better integration with the rest of the TensorFlow library.

\lstinputlisting[language=Python, linerange={82-87}, caption={Layer outs}, label={lst:layer outs}]{source_analysis/utilities.py}