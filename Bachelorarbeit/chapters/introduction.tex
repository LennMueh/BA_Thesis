Deep Neural Networks (DNNs)\cite{lecun_deep_2015} are increasingly used in various applications, including highly sensitive domains such as medical diagnosis\cite{litjens_survey_2017} and autonomous vehicles\cite{bojarski_end_2016}.
DNNs are highly advanced in areas such as image\cite{krizhevsky_imagenet_2012,litjens_survey_2017,ciresan_multi-column_2012}, video\cite{jiang_exploiting_2018}, and speech recognition\cite{hinton_deep_2012}, with a wide range of applications from simple recognition to end-to-end learning\cite{bojarski_end_2016}.

Ensuring the safety of users and others encountering DNNs in safety-critical domains is of utmost importance.
To achieve this, it is necessary to ensure the proper operation of neural networks and identify and repair any bad actors.

This thesis explores the mutation of neurons, identified through spectrum-based fault localization (SBFL), as a potential method for achieving this.
Although SBFL was first successfully used in Eniser et al.'s DeepFault\cite{eniser_deepfault_2019}, where an input was synthesized guided by the suspiciousness values elicited by the SBFL. We aim to take it to a deeper level by mutating the weights and biases based on these suspiciousness values.
It is important to maintain a clear and concise writing style, avoiding complex terminology and sprawling descriptions.
Additionally, we should adhere to conventional academic structure and maintain a formal register, avoiding biased or emotional language.
Finally, we must ensure grammatical correctness and avoid introducing new content.

To adapt DeepFault's SBFL to the current versions of TensorFlow and other frameworks, we first added the necessary functions for weight and bias mutation based on ranked neuron locations.
We then implemented an algorithm to automate the mutation process with just the model, dataset, and desired parameters.
The algorithm can not only merely be used to mutate the not properly working neurons but also mutate them and then be used to train the network.